%%%%%%%%%%%%%%%%%%%%%%%%%%%%%%%%%%%%%%%%%%
 %Copyright (C) 2018-2020 YuZJ.
%使用CC-BY-NC-SA授权。一份完整版本的许可证已位于附录。这个版本原始作者YuZJ,
%邮箱theafamily@126.com(最后连接于2019年06月20日17:32:17)。
%%%%%%%%%%%%%%%%%%%%%%%%%%%%%%%%%%%%%%%%%%
\chapter{GNU宣言}
GNU宣言\footnote{我将这份历史文件添加进本技术指南的目的,是为了让程序员在商业化如潮水般推进时,能够关注一些其它理念。该份文件仅代表自由软件运动捍卫者的观点,并不能代表YuZJ Lab的观点或立场。\cite{gnum},被引用时文章内所有链接均有效。注意,有些脚注是由GNU CTT加的。}(如下所示)由\href{http://www.stallman.org/}{Richard Stallman}在1985年撰写,用来请求大家支持GNU操作系统的开发。其部分文本摘自1983年撰写的初始声明。直到1987年,因为开发的原因它时时小有更改;那时起,看起来最好是保持它不再改变。\par
时过境迁,我们认识到使用不同的措辞可以避免一些常见的误解。从1993年起,我们添加了脚注来澄清这些问题。\par
如果你想安装GNU/Linux系统,我们建议你使用\href{http://www.gnu.org/distros}{100\%自由的GNU/Linux发行版}之一。如果你想做出贡献,请参看\url{http://www.gnu.org/help/help.html}。\par
GNU工程是自由软件运动的一部分,该运动旨在\href{http://www.gnu.org/philosophy/free-sw.html}{捍卫软件用户的自由}。把GNU和“开源”一词联系在一起是错误的—该词汇是1998年由一些不赞同自由软件运动之道德价值的人士发明的。他们使用该词汇来推动同一领域的\href{http://www.gnu.org/philosophy/open-source-misses-the-point.html}{非道德方案}。\par
\section{GNU为何?GNU并非UNIX!}
GNU,代表的是Gnu's NotUnix(GNU并非UNIX),是我正在编写的一个完全兼容Unix的软件系统,这样我就可以把它自由地交给想要使用它的人。\footnote{此处用词不当。其初衷是人们不必为使用GNU系统而支付许可费。但是用词却没有清楚地说明此事,而人们经常理解为这是说GNU的拷贝总是免费或廉价地发行。这不是本意;后来,宣言指出公司提供有偿发行服务的可能性。之后,我也了解到认真区别自由中的“free(自由)”和价格中的“free(免费)”。自由软件是用户有自由修改和发布的软件。有些用户可能得到免费拷贝,而有些用户付费得到拷贝—如果这些资金帮助到软件的改善,善莫大焉。重要的一点是拥有拷贝的用户有自由和其他人一起使用自由软件。}还有几个志愿者在帮助我。我们非常需要大家在时间、金钱、程序和设备方面的贡献。\par
目前,我们有一个可以用lisp编写编辑命令的Emacs文本编辑器、一个源代码级别的调试器、一个兼容yacc的分析器生成工具、一个链接器和大约35个应用程序。shell(命令解释器)也接近完成。一个新的可移植的优化C编译器已经可以自我编译,可能会在年内发布。现有一个初始的内核,不过还需要增加很多功能才可以模拟Unix。当内核和编译器完成后,我们就有可能发布一个适合开发程序的GNU系统。我们会使用Tex作为文本排版工具,不过nroff还需要一些工作。我们还会使用自由的、可移植的XWindow系统。此后,我们还会加入一个可移植的CommonLisp、一个Empire游戏、一个电子表格和数百个应用以及在线文档。最终,我们希望提供Unix系统常规带有的一切有用之物,以及更多。\par
GNU将能够运行Unix的程序,但是它不完全和Unix一样。我们会根据我们在其他操作系统上的感受做出所有合理的改进。特别地,我们计划使用更长的文件名、文件版本号、防崩溃的文件系统、也许带有文件名填充、终端无关的显示支持、最后可能有一个基于Lisp的窗口系统,此时Lisp程序和普通Unix程序可以共享一个屏幕。C和Lisp都将作为系统编程语言。我们会支持UUCP、MITChaosnet和Internet等通信协议。\par
GNU最初的目标是68000/16000之类的带虚拟内存的机器,因为它们是最容易跑起来的机器。让GNU在更小的机器上运行的额外努力就留给那些需要使用这些机器的人吧。\par
为了避免可怕的混淆,请在指示本工程时,发出“GNU”中g的音。
\section{为什么我必须编写GNU}
我认可的黄金法则是如果我喜欢一个程序,我就必须把它分享给喜欢它的人。软件销售商通过让每个用户保证不和其他人分享来分化用户并控制他们。我拒绝以这种方式打破和其他用户组成的统一体。我的良知让我无法签署这样的保密协议或软件许可证协议。几年来,我在人工智能实验室都在反抗这种趋势以及其他冷漠,但是最终他们还是走得太远了:我无法再呆在一个为我做违背我意愿之事的机构。\par
为了能够继续不失颜面地使用计算机,我决定把一些必要的自由软件集合在一起,这样我就能够继续下去而不需要任何非自由软件。我从人工智能实验室辞了职,这样就可以在我发布GNU时避免和MIT产生法律纠葛。\footnote{2.“赠送”是另一个不妥的表达,它再次说明我那时还没有清楚地分开价格和自由的问题。我们现在建议在谈论自由软件时避免这一表达。请参看\href{http://www.gnu.org/philosophy/words-to-avoid.html\#GiveAwaySoftware}{“不清楚的词汇和短语”}了解更多解释}\par
\section{为什么GNU将会兼容Unix}
Unix并不是我理想中的系统,但是它还不算太差。Unix的主要功能看来是好的,而我认为我可以在不破坏这些好功能的情况下填补Unix缺少的东西。而且和Unix兼容可以让许多人能够方便地接纳它。\par
\section{如何获取GNU}
GNU不属于公共领域。GNU允许任何人修改和再发布,但是任何发布者都不能限制它的继续发布。就是说,它不允许专有性的修改。我想让GNU的所有版本都保持自由。\par
\section{为什么许多程序员想要提供帮助}
我发现许多程序员看到GNU很兴奋并想要提供帮助。\par
许多程序员对系统软件的商业化并不高兴。这可能使他们赚到更多的钱,不过这一般要求他们和其他程序员之间是对立关系,而不是伙伴关系。程序员之间的友谊的基本方式是分享程序;而现在典型的市场活动基本上是禁止程序员互相成为朋友。软件买家必须在友谊和守法之间抉择。自然地,许多人认为友谊更重要。但是许多守法的人通常会感到选哪个都不自在。他们变得愤世嫉俗并且认为编程只是一个挣钱的手段。\par
开发GNU和使用GNU而不是专有软件,我们就能够变得友善并守法。另外,GNU成为一个激励和团结其他人加入分享行列的榜样和旗帜。这给予我们一种和谐的感觉,它是使用非自由软件不可能有的。就和我讨论过的程序员来说,大约一半人认为这是一个重要的幸福感,而它是金钱无法替代的。\par
\section{你该如何做出贡献}
(现今,软件帮助任务请看\href{http://fsf.org/campaigns/priority-projects}{高优先级项目列表}和\href{http://savannah.gnu.org/people/?type_id=1}{GNU帮助需求列表},这是GNU软件包的一般任务列表。其他帮助,请看\href{http://www.gnu.org/help/help.html}{帮助GNU操作系统的指南}。)\par
我请求计算机制造商捐助机器和金钱。我请求个人捐助程序和作品。\par
如果你捐助机器,你可以期待的结果就是GNU将会早一天在该机器上运行。捐助的机器应该是完备的、可用的系统,它应该适用于居家的环境,并无需复杂的冷却或供电系统。\par
我已经找到相当多的程序员,他们热切地想要为GNU贡献闲暇时的工作。就大多数项目而言,这些工作很难协调;这些独立完成的部分凑在一起会不工作。但是就替代Unix的特定任务而言,没有这个问题。一个完整的Unix系统包含数百个应用程序,每个都有独立的文档。大多数的接口规格都由Unix兼容性所限定。如果每个贡献者能够编写一个单一的兼容性Unix应用,并使之在原始的Unix系统中正常工作,那么这些应用放在一起就会正常工作。即使出现一些意外的墨菲问题\footnote{1.Murphy,墨菲效应。是指事情如果有变坏的可能,不管这种可能性有多小,它总会发生。},联合这些部件也是可以完成的任务。(内核将需要更密切的沟通,它将会由一个小的、紧凑的小组来进行。)\par
如果我得到金钱上的捐助,我也许能够雇佣一些全职或兼职的人。薪水按照程序员的标准来看的话不高,但是我要找的人要和看重金钱一样看中社区精神的建设。我把这当作一种方法,它让一些人能够全身心地为GNU工作而不用寻求其他谋生的手段。
\section{为什么所有计算机用户都会受益}
一旦GNU完成,任何人都能够自由地得到一个好用的系统,正如得到空气一样。\footnote{这是又一个我没有认真区别“free”一词的两种意思的地方。该陈述并没有错—你是可以免费获得GNU软件,从朋友那里或从网上下载。但是它在提倡错误的理念。}\par
其意义远远超出了只是为每个人省去一份Unix许可证费用。这意味着避免了大量重复的系统编程工作造成的浪费。这些努力就可以用于推进技术的进步。\par
完整的系统资源将向每个人开放。其结果是,如果有用户需要更改系统,他总可以自由地自己修改或雇用其他程序员或公司来改。用户就用不再祈求拥有源代码的那一家公司或那一个程序员来帮他修改,没有人再处于独断的地位。\par
通过鼓励学生学习和改进系统代码,学校能够提供多得多的教育环境。哈佛大学的计算机实验室曾有一个政策:如果程序的源代码不能公开显示在屏幕上,那么就不能安装该程序,这就是坚持拒绝安装某些程序。我受此启发良多。\par
最后,考虑谁是系统软件的所有者以及谁应该做或不做什么的开销也被化解了。\par
筹划人们为一个程序付费,包括许可证费用,因为要通过麻烦的机制来搞清楚一个人应该为该程序支付多少费用,总是会导致大量的社会成本。而且只有管制的国家才能强制每个人都遵守付费制度。举例来说,空间站的空气要花大量成本来制造:为每次呼吸的容量计费是公平的,但是时时都带着测量面具即使是对负担得起呼吸费用的人也是无法忍受的事。加之随处可见的、监控人们是否脱掉面具的摄像头也令人无法容忍。所以,支持空气工厂的最好办法还是只收人头税并摆脱掉面具。\par
复制全部或部分程序对程序员来说和呼吸一样自然,一样有生产力。它也应该一样自由。\par
\section{一些容易驳斥的、反对GNU目标的观点}
\textbf{“如果免费,就没有人会用了,因为用户没有可靠的技术支持。”}\par
\textbf{“你必须对程序收钱才能提供技术支持。”}\par
如果人们宁愿免费获得没有服务的GNU,而不是付费给GNU获得服务,那么为免费GNU提供技术服务的公司应该是有利可图的。\footnote{现在就有几个这样的公司。}\par
我们必须区别对待真正的编程和仅仅是手把手服务这两种形式的技术支持。前者是你不能依赖一个软件供应商来解决的。如果你的问题没有被足够多的人共同体会,那么供应商会告诉你:快走开。\par
如果你的业务需要依赖于技术支持,那么唯一的办法是拥有所有必要的源代码和工具。然后,你就可以雇佣任何有能力的人为你解决问题;你就不必祈求某个特定的人。对Unix,源代码的价格使大多数人都不会考虑。对GNU,这就简单了。还会有找不到能人的时候,但这个问题不是发行策划的问题。GNU并没有解决世界上所有的问题,只是其中一些问题。\par
同时,对计算机知之甚少的用户需要手把手服务:为他们做些很容易但他们真的不知道怎么做的事。\par
这些服务可以由那些只销售手把手服务和修复服务的公司提供。如果用户愿意花钱买带服务的产品,那么他们也应该会为免费的产品购买服务。服务公司竞争的是质量和价格;用户不会绑定在某个服务商上。同时,像我们这样的不需要服务的人可以不用购买服务来使用程序。\par
\textbf{“不打广告,不可能有很多人知道,所以你必须对程序收费才能够支付广告费。”}\par
\textbf{“对免费可得的程序打广告是做无用功。”}\par
有很多免费或极其廉价的宣传形式可以用来通知计算机用户关于GNU的消息。但是使用广告可能会通知到更多的计算机用户。如果真是这样,那么通过广告收费寄送GNU拷贝的业务应该可以赚回广告费及更多。这样的话,只有从该广告获利的用户才付费。\par
另一方面,如果许多人从朋友处获得GNU,而此类业务并不成功,那么说明靠广告传播GNU并无实际必要。为什么自由市场的倡导者不能让自由市场决定这件事呢?\footnote{虽然它不是公司而只是慈善机构,自由软件基金会有10年是靠发行服务来获得其大部分资金的。你可以通过\href{http://www.gnu.org/order/order.html}{从FSF订购东西}来支持它的工作。 }\par
\textbf{“我公司需要专有操作系统来在竞争中取胜。”}\par
GNU将把操作系统软件从竞争的王国中移除。你不能在此取胜,你的对手也不能。你们将在其他方面竞争,但同时在操作系统领域获利。如果你的业务是销售操作系统,那么你不会喜欢GNU,但这对你来说是困难的事。如果你的业务是其他,GNU能够把你从昂贵的操作系统售价中解救出来。\par
我很想看到许多制造商和用户会捐助GNU的开发,这样会降低他们的花费。\footnote{一组公司在1991年左右集资来支持GNU C编译器的维护。}\par
\textbf{“难道程序员不该因为他们的创造力得到回报吗?”}\par
值得回报的东西应该是对社会的贡献。创造力可以是一种社会贡献,但只有在社会能够自由使用其结果时才是。如果程序员应该由于创新程序而得到回报,同理,他们也应该由于限制程序的使用而得到惩罚。\par
\textbf{“难道程序员不能为自己的创造力要求回报吗?”}\par
工作获得报酬或追求更高的薪酬并没有什么不对,只要我们不使用破坏性的手段。但是今天,软件领域的常规手段就是建立在破坏之上的。\par
因为限制减少了程序使用的方法和人数,所以通过限制程序的使用来从用户身上榨取钱财是破坏性的。它限制了人类可以从该程序中获得财富的总量。当限制是故意为之,伤害的结果就是故意破坏。\par
优秀公民不会使用这种破坏手段来致富的原因是,如果每人都这样,我们都会被相互破坏搞得更穷困。这是康德伦理\footnote{Kantian Ethics,康德伦理。是指德国哲学家康德的义务论伦理思想,其基本观点是,世界上只有一个东西是无条件的善,不但它自身是无条件善的,而且也是使一切其他东西成为善的条件,这个东西就是理性,即善良意志。};或者叫黄金定律。因为我不喜欢这样的结果,所以如果每个人都囤积信息,我就有义务说这样做是不对的。特别地,希望个人的创造力有回报并不能证明剥夺其他人的这种创造力就是对的。\par
\textbf{“程序员不就饿死了吗?”}\par
我可能会回答没人被迫成为程序员。我们大多数人无法靠沿街乞讨过活。但结果是,我们并没有被迫沿街乞讨并挨饿。我们会去做其他事情。\par
然而,这个回答是错的,因为它承认了提问者隐含的假设:没有软件的所有权,程序员就可能不会收到任何报酬。据此,报酬不是全部、就是没有。\par
程序员不被饿死的真正原因是他们还有能从编程谋生的方法;只是不如现在赚得多罢了。\par
限制拷贝不是软件行业唯一的基础。它是最常见的基础\footnote{我觉得我说专有软件是软件行业最常见的赚钱基础是个错误。看起来,定制软件开发过去和现在实际上都是最常见的商业模式。这个商业模式不提供收取租金的可能性,所以它必须不断地做事来维持收入。在自由软件的世界,软件定制行业还会继续存在,基本没什么变化。因此,我不再预期程序员在自由软件的世界里收入会变少。}因为它收获了最多的金钱。如果它被禁止或被客户拒绝,软件行业会迁移到那些现在不常用的基础结构之上。总是有多种方式来组织经营活动的。\par
也许在新基础之上的编程工作不再象现在一样可以赚大钱。可是那并不是反驳该变化的论据。现在销售人员按劳取酬并无不妥。如果程序员这样,那么也是正当的。(实际上,他们也许还能赚更多。)\par
\textbf{“难道人们没有权利控制自己的创造力如何被使用?”}\par
“控制自己想法的应用”真的构成对其他人生活的控制;而且通常是使他人的生活更困难。\par
认真研究过知识产权问题\footnote{在20世纪80年代,我还没有意识到谈论“知识产权”的“问题”多么令人困惑。该术语明显是倾向性的;较不明显的事实是,它把针对非常不同问题的多种互不相干的法律纠结在一起。现在,我敦促人们彻底拒绝“知识产权”这一术语,免得它导致其他人以为这些法律构成一个相关的问题。明确的方法应该是独立讨论专利、版权和商标。请参看关于该术语如何散布混乱和偏见的\href{http://www.gnu.org/philosophy/not-ipr.html}{进一步解释}。}的人(比如律师)会说知识产权并非天生的权利。政府确认的那些知识产权种类是有具体目的的特定法律活动的产物。\par
比如,专利体系是为了鼓励发明家公开其发明详情而建立的。其目的是帮助社会而不是帮助发明家。那时,17年的专利期相比技术进步的速度是短暂的。由于专利只是制造商之间的问题,对他们来说,专利协议的花费比生产建设要小,所以专利通常没有太大的害处。专利没有限制使用它们的大多数用户。\par
版权的概念在古代并不存在,那时作者们经常互相大量拷贝非文学类作品。这是很实用的活动,也是许多作者的作品能够哪怕只有一部分流传下来的唯一方法。版权系统为鼓励作者权益而特意创建。在其创建的发明领域—书籍,只有用印刷机才能有效拷贝—版权没什么害处,也没有限制大多数读者。\par
所有知识产权都只是社会发放的许可证,因为人们曾经认为,不管是对还是错,发放这样的许可证可以使整个社会受益。但是就任何具体情况来说,我们都要问:发放该许可证真的让我们受益了吗?获得授权的人能够从事什么活动呢?\par
今天的软件和一百年前的书籍有很大的不同。软件最容易的拷贝是人传人,软件有源代码和目标代码两种不同形式,软件是来使用而不是阅读和欣赏的,这些事实结合在一起就构成了一种情形。在此情形下,加强版权对整个社会在物质和精神上都是伤害;无论法律是否允许,我们此时都不应该再维护版权。\par
\textbf{“竞争使东西变得更好。”}\par
赛跑是竞争的典范:通过回报优胜者,我们鼓励人们跑得更快。当资本主义真的这样运作时,它做得很好;但是其辩护者做的这个假设并不总是对的。如果竞争者忘记了回报的原因而只想着胜利,不计方法,那么他们就可能使用其他的策略—比如攻击别的竞争者。如果竞争者在互相打架,大家就都跑不快。\par
专有软件和保密软件在道德上等同于互相打架的竞争者。令人沮丧的是,我们唯一的裁判看来并不反对打架;他只是规范打架者(“每跑10米,你们可以打一下”)。他真的应该把他们分开,并严惩试图打架的竞争者。\par
\textbf{“没有金钱刺激,人们不就不再编程了吗?”}\par
实际上,许多人在绝对没有金钱刺激的情况下也会编程。编程对一些人有不可抗拒的魔力,这些人往往是最擅长编程的那些人。从来也不缺少坚持音乐的职业音乐家,即使他们毫无希望靠音乐谋生。\par
但是这个问题,虽然经常被问到,并不是指这种情况。程序员会得到报酬,只是变少。所以问题应该是,金钱减少时,还有人编程吗?我的经验是:有。\par
10多年来,许多世界上最好的程序员在人工智能实验室工作,这里的收入要比他们到其他地方工作少得太多。他们获得了许多非金钱的回报:比如,名望和感谢。而创造力本身也是快乐,也是回报。\par
然后,当有机会做同样有趣的工作并赚大钱时,大多数人离开了。\par
这说明人们会为致富之外的理由编程;如果有同时也能赚到大钱的机会,他们也会选择它。薪水低的企业在和薪水高的企业竞争时表现不佳,但是如果薪水高的企业被禁止,低薪水的企业不应该再表现差劲吧。\par
\textbf{“我们迫切需要程序员。如果他们要求我们不要帮助友邻,我们不得不那样做。”}\par
你永远也不会绝望到去遵守这样的命令。请记住:宁为玉碎,不为瓦全!\footnote{Millions for defense, but not a cent for tribute!原意是宁可战斗,也不乞和!}\par
\textbf{“程序员也需要谋生啊。”}\par
短期来看,是这样的。不过,程序员有很多不用出卖程序的使用权利就可以谋生的方法。出卖权利现在成了惯例,是因为它带给程序员和生意人最多的钱财,而不是因为它是谋生的唯一手段。如果想要,我们能够轻易找到其他的方法。这里举几个例子。\par
制造商新引进新计算机需要雇人来把操作系统移植到新硬件上。\par
教育培训、手把手服务和维护服务也可能雇佣程序员。\par
有新想法的人可以发布免费软件\footnote{后来,我们了解到要区别“自由软件”和“免费软件”。“免费软件”是指你可以自由再发布的软件,但是你并没有自由来学习和修改其源代码,所以大部分免费软件不是自由软件。请参看\href{http://www.gnu.org/philosophy/words-to-avoid.html\#GiveAwaySoftware}{“不清楚的词汇和短语”}了解更多解释。},并向对此满意的用户寻求捐助,或者是销售手把手服务。我就碰到一些成功这样做的人。\par
需求相关的用户可以组建用户组,并支付会费。用户组就可以和程序公司签约让公司定制组内成员需要的程序。\par
所有开发费用都可以由软件税来支付:\par
假设每个购买计算机的用户都要按价格支付一定比例的软件税。政府可以让诸如NSF \footnote{NSF, National Science Foundation:美国国家科学基金会。}之类的代理使用该税收支持软件开发。\par
但是如果购买者自己向软件开发做了捐助,那么他可以减税。他可以自己选择捐助项目—通常,他会选择他希望能够用到的项目。减税额度最高是免税。\par
税率可以由交税的人投票决定,票的权重可以按大家的应税额来算。\par
\textbf{结果:}\par
计算机使用社群支持软件开发。\par
该社群决定应该支持到什么程度。\par
用户可以根据自己的需要来选择要支持的项目。\par
长远来看,让软件自由是通往富足世界的一小步;在富足世界里,人们不必辛苦工作来谋生。人们在每周10小时的法律活动、家庭咨询、机器人维修和流星观察等规定任务之外,能够自由投入到象编程这样的有趣活动中。那时,就没有必要再以编程为谋生手段了。\par
我们已经把整个社会要维持生产力的工作大大减少了,但是只有很少一部分转化为劳动者的闲暇,因为生产活动需要夹杂很多的非生产活动。其主要原因是官僚主义和对竞争的抗拒。自由软件会大大减少在软件生产领域的生产力流失。我们必须做这件事,为了使技术进步带来的生产力提高能够转化为人们工作的减少。
\chapter{关于本书}
在这里你能读到一些关于本书的信息。
\section{版本历史}
\subsection{试行版}
开始编写:Jan.28,  2019\par
\subsection{第二版}
开始编写:Dec. 19, 2019
\section{我该如何对这个工程做出贡献?}
请将更改的内容或需要反馈的错误写成一个txt文件,它应该像这样子:
\begin{verbatim}
COM2:https://github.com/YuZJLab/COM2
[版本号]
[错误性质]
[详细内容]
##解释:
##[版本号]应为所使用的提交的哈希值(如156ba48adc942a41a219a65752187bcb7981b99b)。
##[错误性质] 应为以下内容中的任意一个:
##[REF]引用错误;侵权
##此时[详细内容]就应该包括具体的内容,在整个PDF文件中的位置,正确的来源
##以及是否可以在修正后继续使用。
##[CONT]内容错误
##此时[详细内容]就应该包括具体的内容,在整个PDF文件中的位置,正确的内容。
##[COMP]编译错误
##此时[详细内容]可省略。
\end{verbatim} \par
例如:
\begin{verbatim}
COM:https://github.com/YuZJLab/COM_Zh_CN
[CONT]
commit 156ba48adc942a41a219a65752187bcb7981b99b
内容略。
\end{verbatim} \par
\section{感谢}
\ovalbox{王景超}:给了我最初灵感的实验高中电教员。
\section{商标确认}
所有商标的所有权归各自的商标所有者。
\section{编译方法}
对于Windows操作系统,请按以下步骤操作:\par
1.确保你的计算机操作系统为Windows7 sp1及以上。安装TeXLive2019,并确保“C:/texlive/2019/bin/Win32/ ”目录(这是默认安装路径)已被添加到PATH环境变量中,安装并更新所有宏包。\par
2.运行Compile.bat(在一般的Windows系统下双击即可),它应位于源码包的根目录内。\par
3.如果你希望手动编译,请在根目录内打开命令提示符,按顺序输入如下命令:
\begin{verbatim*}
xelatex Main
bibtex Main
xelatex Main
xelatex Main
del *.log
del *.aux
del *.bbl
del *.blg
del *.toc
del *.out
del *.fls
del *.fdb_latexmk
del *.syntex.gz
\end{verbatim*}
对于GNU/Linux操作系统,请按以下步骤操作:\par
1.安装TeXLive 2019,并确保“/usr/local/texlive/2019/bin/x86\_64-linux/ ”目录(这是默认安装路径)已被添加到PATH环境变量中,安装且更新所有宏包。\par
2.运行Compile.sh(在目录内打开(伪)终端,输入“./Compile.sh”。可能需要先运行“chmod +x Complie.sh”),它应位于源码包的TeX目录内。\par
3.如果你希望手动编译,请在TeX目录内打开(伪)终端,按顺序输入如下命令:
\begin{verbatim*}
xelatex Main.TeX
bibtex Main
xelatex Main.TeX
xelatex Main.TeX
rm -f *.log
rm -f *.aux
rm -f *.bbl
rm -f *.blg
rm -f *.toc
rm -f *.out
rm -f *.fls
rm -f *.fdb_latexmk
\end{verbatim*}
无论你使用何种编译方式,你最终将会得到Main.PDF。这就是最终的文件。
\section{关于作者}
YuZJ,浙江省慈溪市实验高中2019届学生,电教委员。现就读于浙江大学-爱丁堡大学联合学院生物信息学系。
\chapter{署名-非商业性使用-相同方式共享 3.0中国大陆}
本作品(定义如下)的提供是以适用“知识共享公共许可协议”( 简称“CCPL”或 “许可”)\footnote{来源:【知识共享许可协议法律文本】\url{https://creativecommons.org/licenses/by-nc-sa/3.0/cn/legalcode}(最后访问于2019年6月23日10:47:30)}为前提的。本作品受《中华人民共和国著作权法》以及其他可适用法律的保护。对本作品的使用不得超越本许可协议授权的范围。\par
如你行使本许可授予的使用本作品的权利,就表明你接受并同意遵守本许可协议的所有条款。鉴于本许可为合同,在你接受这些条款和规定的前提下,许可人授予你本许可所包括的权利。
\section{第一条\ 定义} 
\begin{enumerate}
	\item 本作品:指根据本许可协议提供的以任何方式和形式(包括以数字形式)表达之文学、艺术和科学领域的作品,例如:书籍、手册等文字作品;讲课、演讲、讲道及其他同类性质的作品;戏剧或音乐戏剧作品;曲艺作品;舞蹈作品及哑剧作品;配词或不配词的音乐作品;电影作品和以类似摄制电影的方法创作的作品;素描、绘画、书法、建筑、雕塑、雕刻或版画等作品;摄影作品以及以类似摄影的方法创作的作品;杂技艺术作品;实用艺术作品;与地理、地形、建筑或科学有关的插图、地图、设计图、草图及立体的造型作品;以及法律、行政法规规定的其他文学艺术作品。为本许可协议之目的,本协议有关“本作品”的规定适用于表演、录音制品及广播电视节目。 
	\item 原始作者:就文学或艺术作品而言,指创作本作品的自然人或依法视为本作品作者的法人或其他组织。为本许可之目的,下述情形下的自然人、法人或其他组织适用本许可有关“原始作者”的规定:(1)就表演而言,指演员、歌唱家、音乐家、舞蹈家和其他表演、演唱、演说、朗诵、演奏、表现或者以其它方式表演文学、艺术作品或民间文学艺术的人员;(2)就录音制品而言,指首次将表演的声音或其他声音录制下来的自然人、法人或其他组织;(3)就广播电视节目而言,指传播广播电视节目的组织;(4)作者身份不明的,指行使作品著作权(除署名权外)的作品原件所有人(比如出版社)。
	\item 演绎作品:指基于本作品,或基于本作品与其他已存在的作品而创作的作品,例如翻译、改编、编曲或对文学、艺术和科学作品的其他变更,包括以摄制电影的方法对作品的改编,或其他任何对本作品进行改造、转换、或改编后的形式,包含任何可确认为源自原始作品的修改形式。在本许可定义之下构成汇编作品的作品不视为演绎作品。为避免疑义,并为本许可之目的,当演绎对象为音乐作品时,将其依时间序列关系与动态影像配合一致而形成的结果,视为演绎作品。
	\item 汇编作品:指由于对内容的选择和编排具有独创性而构成智力创作的文学、艺术或科学作品的集合,其中本作品以完整且未经修改的形式和另外一部或多部作品组成集合的整体,而各组成作品本身是分开且独立的,例如百科全书、文选、数据汇编作品,以及本条第1项所列作品之外的作品或者标的。在本许可定义之下构成汇编作品的作品不视为演绎作品(定义如上)。
	\item 许可人:指根据本许可提供本作品的自然人、法人或者其他组织。
	\item 你:指以前就本作品没有违反过本许可协议、或曾违反过协议但已获得许可人明示同意、依据本许可行使权利的自然人、法人或者其他组织。
	\item 授权要素:是指许可人所选择的、并标示在本许可文本标题中的下列基本属性:署名、非商业性使用、相同方式共享。
	\item 发行:指以出售或者其他权利移转方式向公众提供本作品或者演绎作品的原件或者复制件。
	\item 公开传播:指公开朗诵本作品以及以任何方式或程序,包括以有线、无线的方式或通过信息网络公开传播本作品的公开朗诵;或向公众提供本作品,使公众可以在自己选定的地点获得本作品;或以任何方式或程序公开表演本作品或向公众传播本作品的表演,包括通过信息网络传播本作品的表演;或以任何方式,包括符号、声音或图像,广播或转播本作品。上述定义包括相关法律规定的“展览”“表演”“放映”“广播”或通过信息网络向公众传播作品等传播方式。
	\item 复制:指以印刷、复印、拓印、录音、录像、翻录、翻拍等方式制作本作品的复制件。
	\item 人身权:指相关法律赋予作者对本作品所享有的发表权、署名权、修改权以及保护作品完整权。
\end{enumerate}
\section{第二条\ 合理使用}
本许可无意削减、限制或约束你基于《中华人民共和国著作权法》或其他相关法律有关著作权保护的限制或例外的规定对本作品的合理使用。
\section{第三条\ 授权}
根据本许可的条款和条件,许可人在此授予你全球性、免版税、非独占并且在本作品的著作权存续期间内均有效的许可,就本作品行使以下权利:
\begin{enumerate}
	\item 复制本作品或将本作品收入一个或多个汇编作品中,以及复制汇编作品中收录的本作品;
	\item 创作和复制演绎作品,但是任何演绎作品,包括任何形式的翻译作品,均需以合理方式清楚地标示、区分或以其他方法表明原始作品已经被修改或变更。例如,翻译作品可以标明“原作品已由英文翻译为西班牙文”,改编作品可以标明“原作品已作修改”;
	\item 发行、公开传播本作品(包括汇编作品中收录的本作品); 
	\item 发行、公开传播演绎作品。
\end{enumerate}
以上权利可在任何现有的或者以后出现的并为可适用的法律认可的媒体和形式上行使。上述权利包括为在其他媒体和形式上行使权利而必须进行技术性修改的权利。许可人在此保留所有未明示授予的权利,包括但不限于第四条第5项所规定的权利。
\section{第四条\ 限制}
第三条的授权须受以下规定的限制: 
\begin{enumerate}
	\item 你在发行或公开传播本作品时,必须遵守本许可协议。在你发行或公开传播的本作品的每一份复制件中,你必须附上一份本许可协议的复制件或本许可协议的网址(Uniform Resource Identifier)。你不得就本作品提出或增加任何条款,从而限制本许可协议或者限制获得本作品的第三方行使本许可协议所赋予的权利。你不得对本作品进行再许可。你必须在你发行或公开传播的每份作品复制件中完整保留所有与本许可协议及免责条款相关的声明。 在发行或公开传播本作品时,你不得对本作品施加任何技术措施,从而限制从你处获得本作品的第三方行使本许可协议授予的权利。本项(第四条第1项)规定同样适用于收录在汇编作品中的本作品,但并不要求汇编作品中除本作品外的其他作品受本许可协议的约束。在创作汇编作品时,若接到任一许可人的通知,你必须按照其要求,在可行范围内删除汇编作品中根据本协议第四条第4项的要求所作的有关原始作者的身份及其他有关原始作品相关信息的标注。在创作演绎作品时,若接到任一许可人的通知,你必须根据其要求,在可行范围内删除演绎作品中根据第四条第4项的要求所作的有关原始作者的身份及其他有关原始作品的相关信息的标注。
	\item 你必须以下述许可条款发行或公开传播演绎作品:(1)本许可协议;(2)与本许可协议具有相同授权要素的后续版本;或者(3)与本许可协议具有相同授权要素的其他司法管辖区的知识共享许可协议或其后续版本(例如:署名-非商业性使用-相同方式共享 3.0 美国)(以上三类协议统称为“可适用的协议”)。在你发行或公开传播的每件演绎作品的复制件中,你必须附上一份“可适用的协议”的复制件或网址。你不得就演绎作品提出或增加任何条款,从而限制“可适用的协议”的规定,或者限制获得演绎作品的第三方行使“可适用的协议”所赋予的权利。在发行或公开传播包含本作品的演绎作品时,你必须在本作品的每一份复制件中完整地保留所有与“可适用的协议”及免责条款相关的声明。在发行或公开传播演绎作品时,你不得对演绎作品施加任何技术措施,从而限制从你处获得演绎作品的第三方行使“可适用的协议”所赋予的权利。本项(第四条第2项)规定同样适用于收录在汇编作品中的演绎作品,但并不要求汇编作品中除基于本作品而创作的演绎作品之外的其他作品受“可适用的协议”的约束。 
	\item 你不得以任何形式行使本协议第三条授予你的权利去谋取或获得商业利益或私人金钱报酬。若交换过程中未涉及任何商业利益或私人金钱报酬,通过数字文件共享方式或其他方式用本作品去交换其他受著作权保护的作品,将不被视为谋取或获得商业利益或私人金钱报酬。
	\item 在发行或公开传播本作品、任何演绎作品或汇编作品时,除非有依据第四条第1项之要求,否则你必须完整保留所有关于本作品的著作权声明,并以适于所使用的媒介或方法的形式提供下述信息:(1)在原始作者的姓名(或笔名)已被提供的情况下,给出该姓名或笔名,或者在原始作者或许可人以许可人的著作权声明或其他合理的方式,指定可以在作品上署名的他方当事人姓名的情况下,指明该他方当事人的名称(“署名人”);(2)在本作品标题已被提供的情况下,给出本作品的标题;(3)在合理可行的范围内,标明许可人指定需与本作品同时出现的网址,除非该网址没有涉及到本作品的著作权声明或者关于本作品的许可信息;(4)若为演绎作品,则依第三条第2项之要求,必须注明演绎作品中使用的本作品的作者姓名和作品名称(例如,“某作者作品的法语译本”,或“基于某作者作品的电影剧本”)。本项(第四条第4项)要求的对作者姓名和作品名称的指明可采取任何合理方式,但在演绎作品或汇编作品中,如果已经指明了演绎作品的所有作者或汇编作品中所有内含作品的作者,那么对本作品名称和作者姓名的指明须同时出现在任何其他作者姓名出现的地方,并至少与对其他作者的指明一样显著。为避免疑义,本条有关标示作者姓名和作品名称之规定,仅适用于前述署名的用途;除非你事先另行取得原始作者、许可人或署名人的书面同意,否则你不得以明示或者默示的方式主张或暗示,你本人或你对作品的使用与原始作者、许可人或署名人有关联或者已获得上述人士的赞助或者支持。
	\item 为避免疑义,针对不同司法管辖区的著作权许可体系作出如下约定:
	\subitem 权利不能放弃的强制许可体系。在那些许可人不能放弃通过任何法定的或强制的许可方案收取许可使用费的权利的司法管辖区,许可人保留因你行使本许可协议授予的权利而向你收取许可使用费的专有权;
	\subitem 权利可以放弃的强制许可体系。在那些许可人可以放弃通过任何法定的或强制的许可方案收取许可使用费的权利的司法管辖区,许可人放弃因你行使本许可协议授予的权利而向你收取许可使用费的专有权;但若你行使本许可协议授予的权利时未遵守本许可协议第四条第3项有关非商业性使用的规定,则许可人保留向你收取本作品许可使用费的权利;
	\subitem 自愿许可体系。在实行著作权自愿许可的司法管辖区,若你行使本许可协议授予的权利时未遵守本许可协议第四条第3项有关非商业性使用的规定,则许可人保留向你收取本作品许可使用费的权利,许可人可以自行或者通过所参加的著作权集体管理组织向你收取本作品的许可使用费。
	\item 除非其他法律法规另有规定,你在复制、发行或者公开表演本作品,或者复制、发行或者公开表演作为任何演绎作品或汇编作品一部分的本作品时,不得歪曲、损害或者以其他方式损害本作品,导致原始作者的名誉或者荣誉受损。
\end{enumerate}
\section{第五条\ 声明、保证和免责}
除非本许可的当事人相互以书面的方式做出相反约定,且在相关法律所允许的最大范围内,否则许可人按其现状提供本作品,对本作品不作任何明示或者默示、依照法律或者其他规定的陈述或担保,包括但是不限于任何有关可否商业性使用、是否符合特定的目的、不具有潜在的或者其他缺陷、准确性或者不存在不论能否被发现的错误的担保。有些司法管辖区不允许排除前述默示保证,因此这些排除性规定并不一定适用于你。
\section{第六条\ 责任限制}
除非属于相关法律所要求的范围,许可人在任何情况下都不对你因本许可或因使用本作品而产生的任何直接损失、间接损失或惩罚性赔偿负责,即使许可人已被告知发生此类损害的可能性。 
\section{第七条\ 许可终止}
\begin{enumerate}
	\item 在你违反本许可协议任何条款时,本许可及其所授予的权利将自动终止。然而,根据本许可从你处获取演绎作品或汇编作品的自然人、法人或者其他组织,如果他们仍完全遵守相关条款,则对他们的许可不会随之终止。即使本许可被终止,第一条、第二条、第五条、第六条、第七条以及第八条仍然有效。
	\item 在上述条款及条件的前提下,此处授予的许可在法定著作权保护期限内有效。即便如此,许可人保留依其他许可条款发行本作品及在任何时候停止发行本作品的权利;但是,许可人的上述权利不能被用于撤销本许可或任何其他在本许可条款下授予的或必须授予的许可,除本条第1项指明的终止外,本许可将保持其完全效力。
\end{enumerate}
\section{第八条\ 其他事项}
\begin{enumerate}
	\item 当你发行、公开传播本作品或其汇编作品时,许可人给获得作品的第三方提供本作品的许可,其条款和条件与你所获得的许可相同。
	\item 当你发行或公开传播演绎作品时,许可人给获得作品的第三方提供本作品的许可,其条款和条件与你所获得的许可相同。
	\item 如因相关法律,本许可的某一条款无效或不能履行,本许可其余条款的有效性和可履行性不受影响。如本许可的当事人未采取进一步措施,此类无效条款应在必要范围内进行最低限度的修改以使其有效和可履行。
	\item 除非当事人书面同意并签字放弃某条款和允许某违约行为,本许可的任何条款或规定都不应被视为已被放弃,或被视为允许此违约行为。
	\item 本许可构成相关当事人与本授权作品相关的全部协议。除已在此处确认的之外,并不存在任何与本作品相关的谅解备忘录、协议或声明。许可人不受你提出的任何附加规定的约束。未经许可人和你双方书面同意,本许可不得更改。
\end{enumerate}
\section{“知识共享”(Creative Commons)声明}
“知识共享”不是本许可协议的一方,对本作品不作任何相关保证。“知识共享”不对你或任何其他方的任何损失负责,包括但不限于与本许可相关的直接损失和间接损失。虽然有上述两点,但如果“知识共享”已明确标识自己为许可人,它将承担许可人的全部权利和义务。\par
除用于向公众表明本作品是依照知识共享公共许可协议(CCPL)授权以外,如未经“知识共享”事先书面同意,任何一方均不得使用“知识共享”(Creative Commons)商标和其他相关商标及标识。任何被允许的使用都必须符合“知识共享”的现行商标使用准则。该准则已在其网站上发布或可应要求随时提供。为避免疑义,本段关于商标的限制性规定不构成本许可之一部分。\par
你可以通过\url{https://creativecommons.org/}(最后连接于2019年6月23日10:57:40)联系“知识共享”。
\chapter{SIL OPEN FONT LICENSE Version 1.1 - 26 February 2007}
Copyright (c) <dates>, <Copyright Holder> (<URL|email>),\par
with Reserved Font Name <Reserved Font Name>.\par
Copyright (c) <dates>, <additional Copyright Holder> (<URL|email>),\par
with Reserved Font Name <additional Reserved Font Name>.\par
Copyright (c) <dates>, <additional Copyright Holder> (<URL|email>).\par
This Font Software is licensed under the SIL Open Font License, Version 1.1.\par
This license is copied below, and is also available with a FAQ at:\url{http://scripts.sil.org/OFL}(最后连接于2019年12月19日04:54:22)\par
\section{PREAMBLE}
The goals of the Open Font License (OFL) are to stimulate worldwide development of collaborative font projects, to support the font creation efforts of academic and linguistic communities, and to provide a free and open framework in which fonts may be shared and improved in partnership with others.\par
The OFL allows the licensed fonts to be used, studied, modified and redistributed freely as long as they are not sold by themselves. The fonts, including any derivative works, can be bundled, embedded, redistributed and/or sold with any software provided that any reserved names are not used by derivative works. The fonts and derivatives, however, cannot be released under any other type of license. The requirement for fonts to remain under this license does not apply to any document created using the fonts or their derivatives.
\section{DEFINITIONS}
``Font Software'' refers to the set of files released by the Copyright Holder(s) under this license and clearly marked as such. This may include source files, build scripts and documentation.\par
``Reserved Font Name" refers to any names specified as such after the copyright statement(s).\par
``Original Version'' refers to the collection of Font Software components as distributed by the Copyright Holder(s).\par
``Modified Version'' refers to any derivative made by adding to, deleting, or substituting -- in part or in whole -- any of the components of the Original Version, by changing formats or by porting the Font Software to a new environment.\par
``Author'' refers to any designer, engineer, programmer, technical writer or other person who contributed to the Font Software.
\section{PERMISSION \& CONDITIONS}
Permission is hereby granted, free of charge, to any person obtaining a copy of the Font Software, to use, study, copy, merge, embed, modify, redistribute, and sell modified and unmodified copies of the Font Software, subject to the following conditions:
\begin{enumerate}
	\item Neither the Font Software nor any of its individual components, in Original or Modified Versions, may be sold by itself.
	\item Original or Modified Versions of the Font Software may be bundled, redistributed and/or sold with any software, provided that each copy	contains the above copyright notice and this license. These can be included either as stand-alone text files, human-readable headers or
	in the appropriate machine-readable metadata fields within text or binary files as long as those fields can be easily viewed by the user.
	\item No Modified Version of the Font Software may use the Reserved Font Name(s) unless explicit written permission is granted by the corresponding	Copyright Holder. This restriction only applies to the primary font name as presented to the users.
	\item The name(s) of the Copyright Holder(s) or the Author(s) of the Font Software shall not be used to promote, endorse or advertise any Modified Version, except to acknowledge the contribution(s) of the Copyright Holder(s) and the Author(s) or with their explicit written
	permission.
	\item The Font Software, modified or unmodified, in part or in whole, must be distributed entirely under this license, and must not be distributed under any other license. The requirement for fonts to remain under this license does not apply to any document create using the Font Software.
\end{enumerate}
\section{TERMINATION}
This license becomes null and void if any of the above conditions are not met.
\section{DISCLAIMER}
THE FONT SOFTWARE IS PROVIDED "AS IS", WITHOUT WARRANTY OF ANY KIND, EXPRESS OR IMPLIED, INCLUDING BUT NOT LIMITED TO ANY WARRANTIES OF MERCHANTABILITY, FITNESS FOR A PARTICULAR PURPOSE AND NONINFRINGEMENT OF COPYRIGHT, PATENT, TRADEMARK, OR OTHER RIGHT. IN NO EVENT SHALL THE COPYRIGHT HOLDER BE LIABLE FOR ANY CLAIM, DAMAGES OR OTHER LIABILITY, INCLUDING ANY GENERAL, SPECIAL, INDIRECT, INCIDENTAL, OR CONSEQUENTIAL DAMAGES, WHETHER IN AN ACTION OF CONTRACT, TORT OR OTHERWISE, ARISING FROM, OUT OF THE USE OR INABILITY TO USE THE FONT SOFTWARE OR FROM OTHER DEALINGS IN THE FONT SOFTWARE.